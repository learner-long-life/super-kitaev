\subsection{Parallelized Phase Estimation}
\label{subsec:ppe}

One of the key components of the registered phase shifting procedure
described in the previous section is the ability to ``pick'' a random
eigenstate $\ket{\psi_k}$ of a unitary operator $U$ and
measure its corresponding eigenvalue (phase) $\phi_k$
with some degree of precision $\delta = 2^{-n}$ and
error probability $\epsilon = 2^{-l}$. As $n$ increases, the phases
generally become closer together, which is why we need exponential precision
to distinguish between them.
Of course, this exactly describes the phase
estimation procedure, a key technique in many quantum algorithms developed
by Kitaev in his derivation of Shor's factoring result \cite{kitaev}.

\begin{displaymath}
U\ket{\psi_k} = e^{2\pi i \phi_k} \ket{\psi_k}
\end{displaymath}

Phase estimation holds some superposition of eigenstates
$\sum_{i} \alpha_i \ket{\psi_i}$
in an $n$-qubit target register, to which it applies repeated measuring
operators $\Lambda(U^{2^k})$
controlled on some $t$-qubit register, which holds an
approximation $\tilde{phi}$ to the real phase $\phi$.
The unitary $U$ is applied in successive powers of two to get
power-of-two multiples of the phase for increased precision.
The error probability of approximating the phase to within a given
precision is given by the following:

\begin{displaymath}
\Pr\left[ | \phi - \tilde{phi} | \ge \delta \right] \le \epsilon
\end{displaymath}

The parameter $t = t(\delta, \epsilon)$ encodes the dependence of the number
of $\Lambda(U)$ measuring operators as a function of our desired
$\delta$ and $\epsilon$.
It varies according to the exact phase estimation procedure
used.

The popular version of phase estimation presented in \cite[nc00],
requires $t$ repeated controlled applications of some unitary
$U$ (and its successive powers as $U^{2^k}$, $0<k<2^t$)
to a target state which holds some superposition of its eigenvectors,
controlled by $t$ bits which will hold the approximation to a corresponding
eigenvalue (phase).
This version requires applying an inverse quantum Fourier
transform (QFT), which is already high-depth and way more inefficient than our
desired quantum compiler.

To achieve our desired low-depth, we can ``parallelize'' the application of
$\Lambda(U)$ by interpreting the
$t = (n+2)s$ control bits as an $n$-bit number $q$ and
apply $\Lambda(A^q)$ only once.
In the Super-Kitaev procedure, $A$ is the addition operator on an $n$ qubit
target register containing $\psi_{n,k}$, so we can
only effectively add the lowest $n$ bits of $q$.
Furthermore, the eigenvalues of $A$ are rational with a fixed
denominator, $\phi_k = k / 2^n$.
To avoid the inverse QFT,
we can do a classical postprocessing step, which we'll mostly skip over
for fairly good reasons, and then we'll do a detailed resource count of
parallelized phase estimation.

\subsection{Classical Postprocessing}

It is now the point to mention that Kitaev's phase estimation procedure
contains a post processing step which is completely classical in
character, in that they involve a measurement. If this measurement is
projective and the outcomes are completely classical, the remaining steps
can be done on our classical computer (recall our quantum coprocessor model),
and the results fed back into our quantum subroutine, registered
phase shifting. Therefore, as long as we can perform these classical
algorithms in polynomial time (which we can), we don't really care
about the equivalent circuit size and depth.

The steps of classical postprocessing, which will determine some of the
parameters in the earlier, quantum part of phase estimation are as follows.

\begin{enumerate}

\item
Estimate the phase and its power-of-two multiples
$2^j \phi_k$ to
some constant, modest precision $\delta''$, where
$0 \le j < (n+2)$. For each $j$, we
apply a series of $s$ measuring operators targeting the state $\ket{\nu}$
controlled on $s$ qubits in the state $(\ket{0}+\ket{1})/\sqrt{2}$,
essentially encoding the $2^j \phi_k$ as a bias in a coin, and flipping the
coin $s$ times in a Bernoulli trial, counting the number of $1$ outcomes,
and using that fraction to approximate the real $2^j \phi_k$.
\item
Sharpen our estimate to exponential precision $1/2^(n+2)$ using the
$(n+2)$ estimates, each for different bits in the binary expansion of
$phi_k$. Multiplication by successive powers-of-two shift these bits
up to a fixed position behind the zero in a binary fraction representation,
where we can use a finite-automata and a constant number of
bits to refine our $O(n)$-length running approximation.
\end{enumerate}

Three things are worth mentioning about the interrelation of the parameters
between these two steps. Since our phases all have a denominator of $2^n$,
there is no need to run the continued fractions algorithm on multiple
convergents, as is the case with period-finding in Shor's factoring algorithm.
Furthermore, the phases are $1/2^n$ apart, therefore it suffices to approximate
the phases to within $1/2^{n+2}$ in order to break ties, which is where
our range for $j$ comes from above.

The number of trials $s$ comes from the Chernoff bound:

\begin{displaymath}
\Pr \left[ | s^{-1}\sum_{r=1}^s v_r - p_* | \ge \delta'' \right]
\le 2e^{-2\delta'^2 s}
\end{displaymath}

Setting this equal to the desired error probabiliy $\epsilon$ we get

\begin{displaymath}
s = \frac{1}{2\delta''^2}\ln \frac{1}{\epsilon}
\end{displaymath}

We are actually estimating the values $\cos(2\pi \cdot 2^j \phi_k)$ and
$\sin(2\pi \cdot 2^j \phi_k)$, so if we wish to know $2^j \phi_k$ with
precision $\delta''$, we actually need to determine the $\cos(\cdot)$ and
$\sin(\cdot)$ values with a different precision $\delta'$, lower-bounding
it with the steepest part of the cosine and sine curves.

\begin{displaymath}
\delta' = 1 + cos(\pi - \delta'')
\end{displaymath}

The factor $\frac{1}{2\delta''^2}$ depends on the constant precision with
which we determine our $2^j \phi_k$ values. Since classical time is
cheap and quantum gates are expensive, it makes sense to minimize the number
of trials $s$. The following table shows the corresponding values of $1/(2\delta''^2)$
and $\delta'$ as a function of various choices for $\delta'$.

\begin{center}
\begin{table}
\begin{tabular}{|c|c|c|}
\hline
$\delta''$ & $\delta'$   & $1/(2\delta''2)$\\
\hline
$1/16$     & $0.0019525$ & $131,160$\\
$1/8$      & $0.0078023$ & $  8,213$\\
$1/4$      & $0.0310880$ & $    517$\\
\hline
\end{tabular}
\end{table}
\end{center}

By making our $\delta'$
exponentially worse (doubling it) we are only increasing the range of
$j$ a linear amount (by one). In general, for $\delta'=1/2^l$, we get
a final estimate for $\phi = 2^{m-3}$

However, projective measurements are irreversible, and we may wish to
uncompute the phase estimation procedure to restore our ancilla to
their initialized state and reuse them later on. In that case, the
postprocessing can actually be done on a quantum computer using
reversible gates and without projective measurements. That's why
the authors of \cite{ksv02} go to some care to show that all the classical
postprocessing steps can be done in polynomial-size and logarithmic-depth
circuit.
However, to simplify our analysis, we assume the case
in the previous paragraph, and accept the
loss of $n$ ancilla qubits, After all, we only run phase estimation once
to get our initial $\ket{\psi_{n,1}}$ state.

\subsection{Steps in the Parallelized Phase Estimation Algorithm}

The main steps in parallelized phase estimation as applied to Super-Kitaev
are:

\begin{enumerate}

\item Begin with a $t$-qubit ancilla register initialized to $\ket{0}^{\otimes t}$.
%\textsc{Resources} $= [0,0,0,0,0,t]$

\item Place the $t$-qubit register into an equal superposition by
applying $n$ Hadamard gates.
%\textsc{Resources} $= [0,0,0,n,1,0]$

\item Treat $t$ as $2s$ groups of bits, each encoding an $n$-bit number.
Sum them up out-of-place, retaining only the lowest $n$-bits,
to get the superposition
of all $n$-bit numbers, $1/(\sqrt{2^{n}}) \sum_{i=0}^{2^n-1} \ket{q_i}$.
Call this register $\ket{q_i}$.
%\textsc{Resources} $= ADD-OUT(2s \times n)$

\item Reverse the first step by applying another $n$ Hadamards.
%\textsc{Resources} $= [0,0,0,n,1,0]$

\item Apply the gate $\Upsilon(A)$ to the target $\ket{\nu}$ controlled
on $\ket{q_i}$, which is equivalent to adding all $q_i$ in superposition
(in place).
%\textsc{Resources} $= ADD-IN(2 \times n)$

\item Measure the register $\ket{q_i}$ in the computational basis to get 
a particular value $q$ and collapse the register to $\ket{q}$. All $t=(n-2)s$
now contain classical $0$ or $1$ as outcomes of $(n+2)s$ Bernoulli trials.

\item Read out these outcomes into our classical controller
and perform the postprocessing
described above to get an approximation of $\phi$ with precision $\delta$.

\end{enumerate}
