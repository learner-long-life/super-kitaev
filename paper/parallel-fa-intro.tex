\section{Introduction}
\label{sec:intro}

In \cite{ksv02}, the authors describe the development of parallelized
iteration of finite automata, which they later use in a low-depth, highly-
parallel simulation of quantum gates (quantum compiling). We call this
quantum compiling
scheme the Super-Kitaev procedure in contrast to the normal
Solovay-Kitaev construction as described in \cite{nc00}.
However, in these notes, we describe
this method of parallelized iteration as a useful tool in its own right,
even if you don't care about quantum compiling or quantum computing at all.
Finite automata (FA) are useful for computing all kinds of classical functions,
which is how they are used in Super-Kitaev, albeit in a reversible fashion
with the possibility of feeding in quantum inputs and getting out quantum
outputs, in the most general case.
But the idea of parallelizing FA iteration can also be used in an
efficient classical implementation that could be parallelized on modern
multi-core processors. It also demonstrates that we can achieve an
exponential speedup in depth (from linear depth of normal, serial iteration to
logarithmic depth of parallel iteration) using only a polynomial overhead in
gates required in the circuit model.

These notes will develop our general (quantum) model of finite automata, the actual
sequence of gates required, and resource requirements for circuit size,
circuit depth, and ancillae qubits required. We will then
apply this model to two particular instances of FA-computable functions useful in
Super-Kitaev: adding and phase sharpening. We will calculate the actual resources
required for a wide range of input sizes to give us some intuition about the
asymptotic tradeoffs and multiplicate constants involved. These are normally
neglected in theoretical descriptions, but are of interest to engineers
when it comes time to design and build an actual implementation. We will closely
follow the development in \cite{ksv02}, to which the interested reader is
referred for a more concise, self-contained description of the Super-Kitaev
algorithm and its sub-modules.
