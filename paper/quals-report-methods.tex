\section{Methods}
\label{sec:methods}

The code simulating Solovay-Kitaev and 
Super-Kitaev were written using Python 2.5 and numpy 1.4.1.
Resource requirements were measured by running this code on
a Linux machine with an Intel Pentium 3.2 GHz dual core CPU and 1 GB of RAM.

We calculated resource usage by the two algorithms in approximating a
two-qubit
controlled-rotation gate ($\Lambda(R_k) \in SU(4)$) used in the quantum Fourier
transform that is central to many important quantum algorithms.
Due to space and time limitations, we were unable to obtain the actual
compiled universal gate
sequences needed to implement this beyond modest precision.
Algorithm-dependent constraints are provided below with an indication of
why they are beyond the scope of this work.
Therefore, the estimates provided are upper bounds indicating
the worst-case resource usage, which are nevertheless more concrete than
previously known asymptotic bounds. In actual practice, depending on the gate
to be compiled, the performance may be better.

Errors in individual gates in a circuit sum up according to the following
formula, where $U_i$ indicates our ideal gate, $\tilde{U_i}$ indicates our
compiled, approximated gate, and $\epsilon_i  = || U_i - \tilde{U_i} ||$.

\begin{displaymath}
|| U_1\cdots U_n - \tilde{U_1}\cdots\tilde{U_n} || \le \sum_{i=1}^n \epsilon_i
\end{displaymath}

Therefore, for a circuit like the QFT on $t$-qubits with size $O(t^2)$ gates,
we would like our error for individual gates to scale like $O(1/t^3)$.
For Shor's factoring algorithm on an $L$-bit number, the reduction to
order-finding applies the inverse QFT on $t = 2L + O(1)$ assuming some constant
overall error precision. For the commonly used RSA key size of
$L = 2048 = 2^11$ bits, $t \approx 4096 = 2^12$, and we would like an individual
$\Lambda(R_k)$ gate in QFT$^{\dag}$ to have error on the order of
$1/(2^36)$. Accordingly, we compare resources on error precisions from
$1/2$ to $1/2^40$, expressed as $1 \le \log_2(1/\epsilon) \le 40$. Of course,
we are neglecting the enormous overhead of error correction, the threshold
theorem for single qubit error rates \cite{nc00}, and the implementation error
of our universal gate set $\mathcal{G}$.

In addition to the modifications to the original Super-Kitaev procedure
described in the previous sections, we make several assumptions to
simplify our analysis. We treat CNOT and single-qubit gates as taking
equal time, when in actual implementations (for example, ion traps) the CNOT
operation is much longer. The Toffoli gate, which can be implemented using
6 CNOTs and 10 single-qubit gates, is counted as 16 gates.

For Solovay-Kitaev, we used a maximum sequence length of $l_0=6$ in the
preprocessing step for $SU(4)$, an initial $\epsilon_0 = 1/64$ and a
$c_{approx} = 4\sqrt(2)$, which provided a converging value for the number of
recursion levels required to achieve a given error $\epsilon$. Generating
sequences with longer length would require terabytes of storage and days
of computing time.

For Super-Kitaev, we ignore the precompiling step since for the $\Lambda(R_k)$
gate it generates at most 8 $\Lambda(e^{i\phi})$ gates to compile. The
primary overhead of Super-Kitaev is in generating the initial magic state
$\ket{\psi_{n,1}}$ which is amortized over all $\Lambda(e^{i\phi})$
gates, therefore a per-gate comparison of resource usage would only
improve Super-Kitaev's standing relative to Solovay-Kitaev. We assume that we
are only running a single circuit on our quantum computer, therefore some
ancillae qubits are not uncomputed: the $n$ qubits used to store the
initial $\ket{\psi_{n,1}}$ state and the $t$ qubits which are measured to
do classical postprocessing are left in a non-$\ket{0}$ state. If we wish to
run multiple-circuits using Super-Kitaev, we would at least double the
size of our circuit to uncompute this part as well as adding the gates
needed to perform
the classical postprocessing on our quantum computer (since the state
$\ket{\psi_{n,0}}$ is in superposition).